%% ------------------------------------------------------------------------- %%
\chapter{Conceitos}
\label{cap:conceitos}

Neste capítulo será dado algumas definições e conceitos fundamentais sobre 
ontologias, além de ferramentas e linguagens ainda com respeito a ontologia

%% ------------------------------------------------------------------------- %%
\section{Ontologias}\index{ontologias!definicao}
\label{sec:definicao}

Ontologias são utilizadas hoje em diversas áreas para organizar a informação. 
São encontradas na literatura diversas definições para ontologias, propostas 
para aplicação em diferentes áreas de conhecimento e propostas para a construção 
de ontologias(metodologias, ferramentas e linguagens).

Uma das definições mais conhecidas para ontologias é apresenta por 
\cite{gruber1995toward} que diz que uma ontologia é uma especificação explícita 
de uma conceitualização. O termo conceitualização corresponde a uma coleção de 
objetos, conceitos e outras entidades que se assume existirem em um domínio e 
os relacionamentos entre eles \cite{genesereth1987logical}. Um conceitualização 
é uma visão abstrata e simplificada do mundo que se deseja representar.

Porém essa interpretação por meio de conceitualização é discutida por 
\cite{giaretta1995ontologies} quando afirma que a noção de conceitualização é um 
grupo de relações extensionais descrevendo um {\it estado das coisas} particular, 
enquanto que a noção que temos em mente é uma relação intensional, nomeando algo 
como uma rede conceitual a qual se superpõe a vários possíveis 
{\it estados das coisas}.

Com esta abordagem de aspecto intensional \cite{guarino1998formal}, revê a 
definição de conceitualização a fim de obter uma interpretação mais clara. Ele 
se refere a ontologia como um artefato constituído por um vocabulário usado para 
descrever uma certa realidade, mais um conjunto de fatos explícitos e aceitos 
que dizem respeito ao sentido pretendido para as palavras do vocabulário. Este 
conjunto de fatos tem a forma da teoria da lógica de primeira ordem, onde as 
palavras do vocabulário aparecem como predicados unários ou binários. O 
vocabulário formado por predicados lógicos forma a rede conceitual que confere 
o caráter intensional às ontologias. A ontologia define as regras que regulam a 
combinação entre os termos e as relações.

Uma outra intepretação muito mais simples é dada por 
\cite{borst1997construction}. Para ele uma ontologia é uma especificação formal 
e explícita de uma conceitualização compartilhada. Nessa definição, "formal" 
significa legível para computadores, "especificação explícita" diz respeito a 
conceitos, propriedades, relações, funções, restriçṍes, axiomas, explicitamente 
definidos; "compartilhado" quer dizer conhecimento consensual; e 
"conceitualização" diz respeito a um modelo abstrato de algum fenômeno do mundo 
real.

%% ------------------------------------------------------------------------- %%
\section{Tipos de Ontologias}\index{ontologias!tipos de ontologias}
\label{sec:tipos_de_ontologias}

As ontologias podem ser classificadas de diversas formas, porém neste trabalho
será utilizado a classificação quanto a sua função. \cite{guizzardidesenvolvimento} 
ainda a define em 5 tipos:

\begin{itemize}
    \item \textit{Ontologias Genéricas}
    
    São consideradas ontologias mais "gerais" que descrevem conceitos mais 
    amplos. Pesquisas enfocando ontologias genéricas procuram construir teorias
    básicas do mundo, de caráter bastante abstrato, aplicáveis a qualquer domínio
    (conhecimento em seu sentido filosófico de categorização e linguística).
    
    \item \textit{Ontologias de Domínio}
    
    Descrevem conceitos e vocabulários relacionados a domínios particulares. Este
    é um tipo de ontologia mais comum, que geralmente é utilizada para representar
    um "micro-mundo".
    
    \item \textit{Ontologias de Tarefas}
    
    Descrevem tarefas ou atividades genéricas, que podem contribuir na resolução
    de problemas, independente do domínio que ocorrem. Sua principal motivação 
    é facilitar a integração dos conhecimentos de tarefa e domínio em uma 
    abordagem mais uniforme e consistente, tendo por base o uso de ontologias.
    
    \item \textit{Ontologias de Aplicação}
    
    Descrevem conceitos que dependem tanto de um domínio particular quanto de uma
    tarefa específica. Devem ser especializações dos termos das ontologias de 
    domínio e de tarefa correspondentes. Estes conceitos normalmente correspondem
    a regras aplicadas a entidades de domínio enquanto executam determinada tarefa.
    
    \item \textif{Ontologias de Representação}
    
    Explicam as conceituações que fundamentam os formalismos de representação de 
    conhecimento, procurando tornar os compromissos ontológicos embutidos nestes
    formalismos. Um exemplo desta categoria é a ontologia de frames.

\end{itemize}

%% ------------------------------------------------------------------------- %%
\section{Construção de Ontologias}\index{ontologias!construção de ontologias}
\label{sec:construção_de_ontologias}

Primeiramente para a construção de uma ontologia de domínio, é necessário definir
o seu domínio e escopo. Logo em seguida, ainda é necessário escolher uma 
metodologia, uma ferramenta e uma linguagem que serão utilizadas para definir a 
estrutura da ontologia. 

Nem todas as ontologias possuem a mesma estrutura, mas todas possuem pelo menos 
um dos elementos básicos, como:

\begin{itemize}
    \item \textit{Classes}
    
    Normalmente organizadas em taxonomias, as classes representam algum tipo de
    integração da ontologia com um determinado domínio.
    
    \item \textit{Relações}
    
    Representam o tipo de interação entre os elementos do domínio (classes).
    
    \item \textit{Axiomas}
    
    São utilizados para modelar sentenças consideradas sempre verdadeiras.
    
    \item \textit{Instâncias}
    
    São utilizadas para representar elementos específicos, isto é, os próprios 
    dados da ontologia.
\end{itemize}

%% ------------------------------------------------------------------------- %%

As ontologias podem ser classificadas de diversas formas, porém neste trabalho
será utilizado a classificação quanto a sua função. \cite{guizzardidesenvolvimento} 
ainda a define em 5 tipos:

\begin{itemize}
    \item \textit{Ontologias Genéricas}
    
    São consideradas ontologias mais "gerais" que descrevem conceitos mais 
    amplos. Pesquisas enfocando ontologias genéricas procuram construir teorias
    básicas do mundo, de caráter bastante abstrato, aplicáveis a qualquer domínio
    (conhecimento em seu sentido filosófico de categorização e linguística).
    
    \item \textit{Ontologias de Domínio}
    
    Descrevem conceitos e vocabulários relacionados a domínios particulares. Este
    é um tipo de ontologia mais comum, que geralmente é utilizada para representar
    um "micro-mundo".
    
    \item \textit{Ontologias de Tarefas}
    
    Descrevem tarefas ou atividades genéricas, que podem contribuir na resolução
    de problemas, independente do domínio que ocorrem. Sua principal motivação 
    é facilitar a integração dos conhecimentos de tarefa e domínio em uma 
    abordagem mais uniforme e consistente, tendo por base o uso de ontologias.
    
    \item \textit{Ontologias de Aplicação}
    
    Descrevem conceitos que dependem tanto de um domínio particular quanto de uma
    tarefa específica. Devem ser especializações dos termos das ontologias de 
    domínio e de tarefa correspondentes. Estes conceitos normalmente correspondem
    a regras aplicadas a entidades de domínio enquanto executam determinada tarefa.
    
    \item \textif{Ontologias de Representação}
    
    Explicam as conceituações que fundamentam os formalismos de representação de 
    conhecimento, procurando tornar os compromissos ontológicos embutidos nestes
    formalismos. Um exemplo desta categoria é a ontologia de frames.

\end{itemize}

%% ------------------------------------------------------------------------- %%
\section{Construção de Ontologias}\index{ontologias!construção de ontologias}
\label{sec:construção_de_ontologias}

Primeiramente para a construção de uma ontologia de domínio, é necessário definir
o seu domínio e escopo. Logo em seguida, ainda é necessário escolher uma 
metodologia, uma ferramenta e uma linguagem que serão utilizadas para definir a 
estrutura da ontologia. 

Nem todas as ontologias possuem a mesma estrutura, mas todas possuem pelo menos 
um dos elementos básicos, como:

\begin{itemize}
    \item \textit{Classes}
    
    Normalmente organizadas em taxonomias, as classes representam algum tipo de
    integração da ontologia com um determinado domínio.
    
    \item \textit{Relações}
    
    Representam o tipo de interação entre os elementos do domínio (classes).
    
    \item \textit{Axiomas}
    
    São utilizados para modelar sentenças consideradas sempre verdadeiras.
    
    \item \textit{Instâncias}
    
    São utilizadas para representar elementos específicos, isto é, os próprios 
    dados da ontologia.
\end{itemize}

%% ------------------------------------------------------------------------- %%
\section{Metodologia para Construção de Ontologias}
\index{ontologias!metodologias para construção de ontologias}
\label{sec:metodologias_para_construcao_de_ontologias}

As metodologias para construção de ontologias tem por objetivo organizar e definir
um padrão para sua construção. O problema que ainda não estão suficientemente
maduras e não conseguem demonstrar um processo realmente estruturado a ponto de 
ser considerado um padrão de fato. Por isso, \cite{guizzardidesenvolvimento} 
sugere uma abordagem sistemática para a sua construção. Essa metodologia é 
composta por 6 fases.

A figura abaixo demonstra o diagrama de atividade que ilustra esta metodologia.

\begin{figure}[!h]
  \centering
  \includegraphics[width=.40\textwidth]{fig_metodologia_diagrama} 
  \caption{Diagrama de atividades da metologia proposta por 
    \cite{guizzardidesenvolvimento}. Fonte da imagem \cite{morais2007ontologias}. 
  }
  \label{fig:diagrama_metodologia} 
\end{figure}

\begin{itemize}
    \item \textit{Identificação de Propósito e Especificação de Requisítos}
    
    Tem o objetivo de identificar a competência da ontologia, ou seja, seus usos
    e propósitos, e para atingir isso limita-se o que é relevante para a ontologia
    e o que não é. Nesta fase também identificam-se potenciais usuários da ontologia,
    e o contexto base que motivou sua construção.
    
    Após a definição das competências, ainda é preciso especificar algumas 
    questões que a ontologia deve ser capaz de responder. Essa especificação 
    envolve a descrição do propósito e dos usos da ontologia e servem também para
    justificar a existência da ontologia e para futuras avaliações da mesma.
    
    \item \textit{Captura da Ontologia}
    
    Esta é considerada a fase mais importante, o seu objetivo é capturar o 
    conjunto de elementos de um domínio que podem ser representados em uma 
    ontologia, com base nas questões de competência relacionadas a esta.
    
    Na captura é feita a identificação dos conceitos (classes), seus 
    relacionamentos e todos os demais elementos exigidos na construção de uma
    ontologia, como: axiomas, instâncias e propriedades.
    
    Os conceitos, suas propriedades e relacionamentos, formam a base de qualquer
    ontologia. Entretanto para definir a semântica de seus termos devem ser 
    construídos os axiomas, que são utilizados para modelar sentenças sempre
    verdadeiras. Na prática axiomas especificam definições em linguagem natural,
    mas também podem ser especificados através de lógica de primeira ordem 
    \cite{falbo1998integracao}.
    
    \item \textit{Formalização da Ontologia}
    
    A formalização de uma ontologia diz respeito a especifica-la por meio de uma
    linguagem. Embora uma ontologia possa ser representada por meio de qualquer 
    linguagem, ou seja, formal ou não formal, a linguagem formal por ser baseada
    em um modelo matemático permite que sejam realizadas pressuposições 
    implícitas e com isso testada com maior precisão e facilidade.
    
    \item \textit{Integração com Ontologias existentes}
    
    \cite{falbo1998integracao} destaca que na fase de captura e formalização,
    pode aparecer a necessidade de integrar a ontologia que se esta criando com
    uma outra já existente, e isto deve ser incentivado pois é uma boa prática 
    aproveitar conceituações previamente estabelecidas em outras ontologias, isso
    reduz o trabalho de ter que reinventar todos os conceitos novamente.
    
    \item \textit{Avaliação}
    
    \cite{guizzardidesenvolvimento} sugere que esta fase seja executada em 
    conjunto com as fases de captura e formalização. Esta fase vida garantir 
    que os requisítos definidos antes da construção da ontologia foram atendidos.
    Para isso, pode-se considerar alguns critérios, como: clareza, coerência, 
    extensibilidade e compromissos ontológicos minímos.
    
    \item \textit{Documentação}
    
    Todo o processo de construção da ontologia deve ser documentado, e esta etapa
    inclui os propósitos, requisitos e cenários de motivação, as descrições 
    textuais da conceituação, a ontologia formal e os critérios de projeto 
    adotados.
\end{itemize}

%% ------------------------------------------------------------------------- %%
\section{Ferramentas para Construção de Ontologias}
\index{ontologias!ferramentas para construção de ontologias}
\label{sec:ferramentas_para_construcao_de_ontologias}

O processo de construção de uma ontologia é uma tarefa custosa e complexa, por 
isso existem algumas ferramentas e APIs que permitem manipular ontologias, tanto
quanto a sua construção até consultar e integrá-la com outras aplicações.

A ferramenta que será utilizada neste trabalho é o Protégé, veja a figura 
\ref{fig:fig_protege_apresentacao}. O Protégé foi criado 
no Centro de Pesquisa em Informática Biomédica da Universidade de Stanford 
\cite{protege}, possui código aberto e por meio dele é possível criar e
manipular ontologias. Outras características muito interessante do Protégé, 
é que ele permite escalabilidade e extensibilidade através de uma arquitetura de 
plugins. O protégé também permite que a ontologia criada nele possa ser exportada 
para diversas linguagens, como

\begin{figure}[!h]
  \centering
  \includegraphics[width=.40\textwidth]{fig_protege_apresentacao} 
  \caption{Protége}
  }
  \label{fig:fig_protege_apresentacao} 
\end{figure}













