%% ------------------------------------------------------------------------- %%
\chapter{Introdução}
\label{cap:introducao}

Por meio do portal do STF na área de Jurisprudênia\footnote{http://stf.jus.br/portal/jurisprudencia/pesquisarJurisprudencia.asp}, 
profissionais da área de direito ou qualquer outro cidadão, têm acesso a todos 
os acórdãos que formam a jurisprudência do orgão. A ferramenta de pesquisa permite 
realizar buscas por acórdãos com base em diversos critérios, como, número do 
acórdão, nome do ministro, data, órgão julgador, legislação etc. 

Porém, mesmo com todos estes critérios a busca não é uma tarefa simples, pois 
o direito sempre demandou domínio extensivo da área por parte da pessoa que deseja 
pesquisar algo, e como a busca leva em consideração somente os termos que foram 
informados, os resultados nas buscas, na maiorias da vezes, não retorna 
informação relevante e mesmo depois da busca ainda é necessário um trabalho de 
separar o que realmente é interessante.

Assim, este trabalho se propõe a elaborar uma nova abordagem de consulta, por meio
da construção de uma ontologia de domínio para a base de acórdãos do STF. Além 
disso, isto é esperado como resultado do trabalho buscas mais dinâmicas por meio 
da web semântica.

%% ------------------------------------------------------------------------- %%
\section{Motivação}
\label{sec:motivacao}

A base de dados do STF é composta por um grande volume de acórdãos, e além do que 
já existe, a cada dia mais, soma-se a esta base novos casos a medida que vão sendo
julgados. O mecanismo de busca permite a utilização de vários critérios, mas 
mesmo eles tornam a busca na maioria das vezes engessada, uma vez que trabalha 
considerando somente os termos que foram informados e não utiliza nenhum 
mecanismo de semântica o que poderia dar um maior dinamismo as buscas.

Por isto o desenvolvimento de um sistema de busca baseado em ontologia 
possibilitaria buscas mais dinâmicas, facilitando o processo de pesquisa por 
até mesmo, pessoas que possuem determinado grau de leigo no domínio do direito.

%% ------------------------------------------------------------------------- %%
\section{Objetivos}
\label{sec:objetivo}

Objetivos Gerais:

\begin{itemize}
    \item Construir uma ontologia de domínio de acórdãos do STF por meio da tecnologia OBDA.
    
    \item Implementar um sistema com interface de pesquisa dos acórdãos
\end{itemize}

%% ------------------------------------------------------------------------- %%
\section{Contribuições}
\label{sec:contribucoes}



%% ------------------------------------------------------------------------- %%
\section{Organização do Trabalho}
\label{sec:organizacao_trabalho}

No Capítulo~\ref{cap:conceitos}, apresentamos os conceitos.


